\graphicspath{{./ch_conclusion_and_outlook/figures/}}

\chapter{Conclusions and outlook}
\label{ch:conclusion}

\begin{center} 
    \vspace{-1cm} {M.S. ~Blok} 
\end{center}

Future quantum technologies have the potential to have great impact on the fields of computation, communication and metrology. The work presented in this thesis capitalizes recently developed low temperature control techniques of the NV center in diamond to study quantum measurements and implement real-time feedback protocols. This transition from open-loop control to feedback control experiments will aid the development of diamond-based quantum technologies. In this chapter we give an overview of the main results and conclusions and provide an outlook for future research directions.
\clearpage

\section{Summary}
The results of this thesis can be summarized as follows.
\begin{itemize}

  \item A variable-strength measurement of the nitrogen spin of an NV center can be implemented via the electron spin. The backaction of sequential variable-strength measurements can be used to manipulate the nitrogen spin when digital feedback is incorporated.

  \item The advantage of adaptive frequency estimation protocols using the electron spin of a NV center is that fewer measurements are required to reach the same sensitivity compared to the best non-adaptive protocols. When overhead is included in the analysis this results a more accurate estimation at any fixed measurement time.

  \item Two electron spins in spatially separated diamonds can be entangled by performing a joint measurement of photons originating from the two NV centers. 

  \item Remote entanglement between two NV centers established via an heralded protocol can be used to unconditionally teleport the state of a nuclear spin to a distant electron spin.

  \item Weakly-coupled carbon spins in an isotopically purified diamond can maintain their coherence even after 200 repetitive resets of the electron spin.
  
\end{itemize}

The experiments that are presented in this thesis
\section{Quantum Information Processing with NV centers in diamond}



\bibliographystyle{../thesis}
\bibliography{conclusion}
