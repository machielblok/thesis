\graphicspath{{./ch_conclusion_and_outlook/figures/}}

\chapter{Conclusions and outlook}
\label{ch:conclusion}

\begin{center} 
    \vspace{-1cm} {M.S. ~Blok} 
\end{center}

Future quantum technologies have the potential to have great impact on the fields of computation, communication and metrology. The work presented in this thesis capitalizes recently developed low temperature control techniques of the NV center in diamond to study quantum measurements and implement real-time feedback protocols. This transition from open-loop control to feedback control experiments will aid the development of diamond-based quantum technologies. In this chapter we give an overview of the main results and conclusions and provide an outlook for future research directions.
\clearpage

\section{Summary}
The results of this thesis can be summarized as follows.
\begin{itemize}

  \item A variable-strength measurement of the nitrogen spin of an NV center can be implemented via the electron spin. The backaction of sequential variable-strength measurements can be used to manipulate the nitrogen spin when digital feedback is incorporated.

  \item The advantage of adaptive frequency estimation protocols using the electron spin of a NV center is that fewer measurements are required to reach the same sensitivity compared to the best non-adaptive protocols. When overhead is included in the analysis this results a more accurate estimation at any fixed measurement time.

  \item Two electron spins in spatially separated diamonds can be entangled by performing a joint measurement of photons originating from the two NV centers. 

  \item Remote entanglement between two NV centers established via an heralded protocol can be used to unconditionally teleport the state of a nuclear spin to a distant electron spin.

  \item Weakly-coupled carbon spins in an isotopically purified diamond can maintain their coherence even after 200 repetitive resets of the electron spin.
  
\end{itemize}

This thesis reports the first experiments with spins in diamond where measurement outcomes are used as input for subsequent control operations thereby closing the loop between measurement and control. Furthermore they establish NV centers as a leading platform for building quantum networks.  Other systems are of course being developed in parallel and like spins in diamond each has their advantages and disadvantages. Recent advances include generation of remote entanglement between trapped ions \cite{Moehring_Nature_2007} and atoms\cite{Hofmann_Science_2012,Ritter_Nature_2012} and atomic ensembles\cite{Chou_Nature_2005} and the implementation of real-time feedback protocols using superconducting circuits\cite{Vijay_Nature_2012,Riste_Nature_2013} and photons \cite{Gillett_Phys.Rev.Lett._2010,Sayrin_Nature_2011}. The following sections will provide an outlook for future research directions with NV centers in diamond.

\section{Quantum Information Processing with NV centers in diamond}
A scalable solution for implementing quantum information technology with NV centers remains a long-term goal and making predictions of how this technology can be developed unavoidably contains some speculation. However it is possible to identify short-term challenges and provide possible solutions to overcome them. The demands on the system will depend on the application. A quantum computer will probably require a vast amount of physical qubits, while a big challenge for a quantum internet lies in entangling nodes with a relatively small amount of qubits over large distances. I will first discuss the challenge of locally scaling up the number of qubits and then proceed to connecting remote quantum nodes.

One challenge that can be overcome using real-time feedback is to protect quantum states against errors. A promising way to deal with noisy operations is to redundantly encode a logical qubit in multiple physical qubits and detect errors by measuring joint properties of the physical qubits without destroying the logical state. When an error is detected it can be corrected with real-time feedback based on the outcome of the multi-qubit measurements. This measurement based quantum error correction has recently been implemented using three weakly coupled $^{13}$C-spins and the quantum non-demolition measurement of the electron spin presented in chapter \ref{ch:AMC} to correct for one type of error\cite{Cramer_arXiv_2015}. The next step is to increase the number of encoding qubits to protect against arbitrary errors and to improve the gate fidelities to reach the scalability threshold. Higher gate fidelity could be achieved using asymmetrical dynamical decoupling sequences\cite{Casanova_arXiv_2015} or via numerical optimization \cite{Liu_NatCommun_2013,Dolde_NatCommun_2014}. Using dynamical decoupling spectroscopy techniques it has been demonstrated that six individual carbons can be identified\cite{Taminiau_Phys.Rev.Lett._2012}. It is still an open question how many carbon spins can be controlled with the electron spin. However it seems impractical to control a large amount of carbon spins via a single electron spin because all operations need to be applied sequentially and optimizing gates will become increasingly difficult for larger systems.

A promising way to engineer a scalable system is to adopt a modular approach where individual nodes consisting of one electron spin and a few nuclear spins are entangled using the optical interface. A recent proposal to implement the surface code using this quantum network architecture showed that the error thresholds of the entanglement generation can be reasonably high (10\%) if the local error rates for initialization, control and measurement are in the order of a percent\cite{Nickerson_NatCommun_2013}. For this approach the heralded entanglement protocol as presented in chapter \ref{ch:LDE} would have to be improved, given the current entanglement rate of 1/250 s$^{-1}$. To this end the NV center can be placed in a fiber-based micro cavity to enhance the photon collection efficiency and the emission in the zero phonon line via the Purcell effect\cite{Kaupp_Phys.Rev.A_2013,Albrecht_Phys.Rev.Lett._2013,Janitz_arXiv_2015}. When a highly connected quantum network is realized it would also lend itself for performing measurement-based quantum computing\cite{Raussendorf_Phys.Rev.Lett._2001,Benjamin_Laser&Photon.Rev._2009} where highly entangled graph states are first created and then the computation is performed by adaptive measurements.

To increase the separation between nodes in a quantum network for quantum communication one needs to overcome the optical losses. For photons emitted in the zero phonon line (wavelength 637 nm) the attenuation in an optical fiber is in the order of 12 dB/km. In a recent result entanglement over a distance of 1.3 km between two NV centers has been demonstrated\cite{Hensen_arXiv_2015}. In this experiment the entanglement rate was severely reduced by losses in the fiber. One way to overcome this is to downconvert the photons to telecom wavelength as was recently demonstrated with quantum dot emission\cite{DeGreve_Nature_2012,Zaske_Phys.Rev.Lett._2012}. Furthermore by employing weakly coupled nuclear spins as memories, entanglement purification\cite{Campbell_Phys.Rev.Lett._2008} and quantum repeater protocols\cite{Briegel_Phys.Rev.Lett._1998} can improve the efficiency of probabilistic entanglement generation.

\section{Single spin sensors}
Employing NV centers in diamond as quantum sensors \cite{Taylor_NatPhys_2008,Schirhagl__2014} has gained a lot of interest since the electron spin is sensitive to many physical quantities like as temperature\cite{Acosta_Phys.Rev.Lett._2010,Toyli_PNAS_2013}, strain\cite{Ovartchaiyapong_NatCommun_2014} and electric and magnetic fields\cite{Taylor_NatPhys_2008,Dolde_NatPhys_2011} with very high spatial resolution. There are several methods to bring the NV center close to the sample. The use of shallow NV centers in bulk diamond has enabled the detection of Johnson noise\cite{Kolkowitz_Science_2015} and spin waves\cite{vanderSar_NatCommun_2015} as well as the magnetic field of biological samples\cite{LeSage_Nature_2013}. Alternatively an NV center can be embedded at the end of a sharp tip that is scanned across the sample\cite{Balasubramanian_Nature_2008,Maletinsky_NatNano_2012,Rondin__2012,Pelliccione_Phys.Rev.Applied_2014,Haberle_NatNano_2015}. Finally NV centers in 
 nanocrystals with a size of a few nanometer can be used and it has been shown that they can be inserted into living cells\cite{Kucsko_Nature_2013}. A severe limitation of shallow NV centers is that surface effects can significantly deteriorate the stability of NV centers since nearby charge can lead to a conversion to NV$^-$ and magnetic noise reduces the coherence times of the spins. It was recently shown that a combination of surface treatment and annealing can improve the optical stability of shallow NV centers in bulk diamond\cite{Chu_NanoLett._2014}.

\section{Fundamentals of quantum mechanics}

Bell \cite{Hensen_arXiv_2015}
reality of wavefunction \cite{Pusey_NatPhys_2012}
weak measurements \cite{Aharonov_Phys.Rev.Lett._1988} toin coss \cite{Ferrie_Phys.Rev.Lett._2014} contextuality \cite{Pusey_Phys.Rev.Lett._2014}
gravitational collapse (\cite{Diosi_PhysicsLettersA_1987}, \cite{Penrose_Phil.Trans.R.Soc.Lond.A_1998}) exp with NV (\cite{Wezel_Proc.R.Soc.A_2012})

\bibliographystyle{../thesis}
\bibliography{conclusions}
