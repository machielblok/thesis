\chapter{Summary}

Gaining precise control over quantum systems is crucial for applications in quantum information processing and quantum sensing and to perform experimental tests of quantum mechanics. The experiments presented in this thesis implement quantum measurements and real-time feedback protocols that can help to achieve these goals using single electron and nuclear spins in diamond. Spins associated with the Nitrogen Vacancy (NV) center in diamond recently emerged as an excellent testbed to demonstrate quantum effects and are a promising building block for future quantum technology.\\

The NV center is an atomic defect in the diamond lattice consisting of a substitutional nitrogen atom next to an empty lattice site. With its effective electron spin and nearby nuclear spins it forms a natural multi-qubit register with long-lived spin states that can be manipulated with magnetic resonance techniques. At temperatures below 10 K it displays spin-selective optical transitions that can be individually addressed and thereby provide an optical interface enabling high-fidelity single-shot readout and the generation of spin-photon entanglement.\\

In chapter 3 the fundamental trade-off between information gain and state disturbance associated with a quantum measurement is investigated. A variable strength measurement of the nuclear spin associated with the host nitrogen atom is implemented via an indirect measurement using the electron spin. The measurement strength can be tuned by varying the amount of entanglement between the two spins. To avoid dephasing of the nuclear spin, due to spin-flips of the electron spin during its readout, a dynamical-stop readout is used to perform a QND measurement of the electron spin. This enables sequential partial measurements that can manipulate the nuclear spin using only the backaction of quantum measurements combined with real-time feedback. \\

The electron spin can be used to sense static magnetic fields by performing repetitive Ramsey sequences. The experiments presented in chapter 4 address the open question whether adaptive measurements can out-perform non-adaptive protocols for sensing applications. An adaptive strategy is implemented where the readout basis is optimized in real-time using a Bayesian estimation based on previous measurement outcomes. The experiment shows that this adaptive protocol outperforms the best known non-adaptive protocol when overhead is taken into account.\\

The results in chapter 5 demonstrate the generation of measurement-based entanglement between two electron spins separated by 3 meters. By locally entangling each electron spin with a photon and performing a subsequent joint measurement of the photons, entanglement between the electron spins is heralded. The generated Bell-pair shared between remote locations is then used to unconditionally teleport the state of a nuclear spin in one diamond to the electron spin in the other diamond. To this end the nuclear spin is prepared in the state that is to be teleported followed by a local measurement of the electron and nuclear spin in the Bell-basis. This measurement projects the electron spin in the other diamond to the initial state of the nuclear spin up to a unitary operation that depends on the outcome of the Bell-state measurement. The original state is then recovered via a feed-forward operation. The fact that the protocol to prepare the remote Bell-pair is heralded and that the local Bell-state measurement can distinguish all four Bell-states, allows for unconditional teleportation. \\

The final two chapters of this thesis discuss the use of weakly coupled $^{13}$C spins as a quantum memory that is robust against optical excitation of the electron spin. The ability to store a quantum state in a quantum register while remotely connecting it to other registers enables the implementation of entanglement purification and quantum repeater protocols. A theoretical model is introduced to analyze the dephasing of a carbon spin during repetitive resets of the electron spin (which is required in the presented heralded entanglement protocol). This model is then tested experimentally in an isotopically purified diamond with a carbon spin that is relatively insensitive to perturbations induced by electron spin-flips owing to its low coupling strength (200 Hz). Although the observed dephasing is stronger than predicted by the model the results indicate that it is possible to store a quantum state in the $^{13}$C spin while optically exciting the electron spin.

\chapter{Samenvatting}

Het nauwkeurig controleren van quantum systemen is essentieel zowel voor toepassingen in quantum informatica en quantum sensoren als voor experimentele tests van de quantum mechanica. De experimenten die in dit proefschrift worden beschreven, introduceren quantum metingen en terugkoppeling protocollen met enkele spins in diamant om deze doelen te bereiken. Spins nabij het stikstof-holte (nitrogen-vacancy, NV) centrum in diamant zijn uiterst geschikt om quantum effecten te demonstreren en zijn een veelbelovende bouwsteen voor toekomstige quantum technologie. \\

Het NV centrum is een atomisch defect in het diamant rooster bestaande uit een stikstofatoom in plaats van een koolstofatoom en een ontbrekend koolstofatoom op een naburige roosterplek. Met zijn elektron spin en nabijgelegen kernspins vormt het NV centrum een quantum register bestaande uit spins met lange coherentietijden die gemanipuleerd kunnen worden met magnetische resonantie technieken. Bij temperaturen beneden de 10 K vertoont het NV centrum spin-selectieve optische transities die individueel aangeslagen kunnen worden. Op deze manier kan de electron spin zeer betrouwbaar worden geinitialiseerd, gemeten en verstrengeld met een foton. \\

In hoofdstuk 3 wordt de fundamentele balans tussen het verkrijgen van informatie en het verstoren van een quantum toestand die hoort bij een quantum meting onderzocht. Door de kernspin te meten via de elektron spin kan de sterkte van de meting gevarieerd worden. Hierbij bepaalt de mate van verstrengeling tussen de twee spins de meetsterkte. Een nieuw ontwikkelde QND meting van de elektron spin kan decoherentie van de kernspin tijdens de uitlezing van de elektron spin verminderen. Dankzij deze QND meting kunnen opeenvolgende gedeeltelijke metingen op de kernspin worden gedaan. Tot slot laten we zien dat de terugslag van opeenvolgende gedeeltelijke metingen gebruikt kan worden om de toestand van de kernspin te manipuleren door gebruik te maken van tergkoppeling.\\

De elektron spin kan worden gebruikt om statische magnetische velden te meten door opeenvolgende Ramsey sequenties uit te voeren. De experimenten die besproken worden in hoofdstuk 4 richten zich de open vraag of het gebruik van terugkoppeling kan helpen om een quantum sensor te verbeteren. In het gedemonstreerde protocol met terugkoppeling wordt de meetbasis van de elektron spin geoptimaliseerd door een Bayesiaanse schatting te maken van het magneetveld gebaseerd op de voorgaande meetuitkomsten. De experimenten tonen aan dat het gebruik van terugkoppeling voordelig is wanneer men de \textit{overhead} meerekent.\\

De resultaten in hoofdstuk 5 demonstreren dat twee elektronen spins in diamanten op een afstand van 3 meter met elkaar verstrengeld kunnen worden door middel van metingen. Door de individuele elektronen spins lokaal te verstrengelen met een foton en vervolgens de fotonen gezamenlijk te meten, worden de elektronen spins in een verstrengelde toestand geprojecteerd. Het resulterende Bell-paar dat gedeeld wordt tussen de twee locaties wordt gebruikt om de toestand van een kernspin in de ene diamant te teleporteren naar een elektron spin in de andere diamant. Hiertoe wordt de kernspin in de te verzenden toestand gebracht waarna de kernspin en de elektron spin lokaal gemeten worden in de Bell-basis. Door het resultaat van deze meting via een klassiek kanaal te communiceren naar de ontvangst locatie kunnen we met terugkoppeling de elektron spin in de andere diamant in de gewenste toestand brengen. \\

In de laatste twee hoofdstukken van dit proefschrift wordt de mogelijkheid onderzocht om zwak gekoppelde $^{13}$C spins te gebruiken als een quantum geheugen dat robuust is tegen verstoringen die kunnen optreden wanneer de elektron spin optisch aangeslagen wordt. Het vermogen om een quantum toestand op te slaan in een quantum register terwijl het tegelijkertijd verstrengeld wordt met een ander register maakt het mogelijk om \textit{entanglement purification} en \textit{quantum repeater} protocollen te implementeren. In een theoretisch model wordt de decoherentie van de koolstof spin beschreven wanneer de elektron spin herhaaldelijk geinitialiseerd wordt (net zoals in het hierboven genoemde protocol om elektronen over afstand te verstrengelen). Dit model wordt getest in een diamant met relatief weinig $^{13}$C isotopen waarin de koolstof spins dankzij hun lage koppelingssterkte (200 Hz) minder gevoelig zijn voor verstoringen door de elektron spin. Hoewel de waargenomen decoherentie sterker is dan het model voorspelt, laten de resultaten zien dat het mogelijk is om een quantum toestand op te slaan in een $^{13}$C spin terwijl het elektron optisch aangeslagen wordt.


