\chapter{Summary}

Gaining precise control over quantum systems is crucial for developing quantum technologies like quantum information processing and quantum sensing and to perform experimental tests of quantum mechanics. The experiments presented in this thesis implement quantum measurements and real-time feedback protocols that can help to achieve these goals using single electron and nuclear spins in diamond. Spins associated with the Nitrogen Vacancy (NV) center in diamond recently emerged as an excellent testbed to demonstrate quantum effects and are a promising building block for future quantum technology.\\

The NV center is an atomic defect in the diamond lattice consisting of a substitutional nitrogen atom next to an empty lattice site. With its effective electron spin and nearby nuclear spins it forms a natural multi-qubit register with long-lived spin states that can be manipulated with magnetic resonance techniques. At temperatures below 10 K it displays spin-selective optical transitions that can be individually addressed and thereby provide an optical interface enabling high-fidelity single-shot readout and the generation of spin-photon entanglement.\\

In chapter 3 the fundamental trade-off between information and disturbance associated with a quantum measurement is investigated. A variable strength measurement of the host nitrogen spin is implemented via an indirect measurement using the electron spin. Here the measurement strength can be tuned by varying the amount of entanglement between the two spins. To avoid dephasing of the nuclear spin due to spin-flips during the electron spin readout a dynamical-stop readout is used to perform a QND measurement of the electron spin. This enables sequential partial measurements that can manipulate the nuclear spin using only the back-action of quantum measurement combined with real-time feedback. \\

The electron spin can be used to sense static magnetic fields by performing repetitive Ramsey sequences. The experiments presented in chapter 4 address the open question whether adaptive measurements can out-perform non-adaptive protocols for sensing. An adaptive strategy is implemented where the readout basis is optimized in real-time using a Bayesian estimation based on previous measurement outcomes. The experiment shows that this adaptive protocol outperforms the best known non-adaptive protocol when overhead is taken into account.\\

The results in chapter 5 demonstrate the generation of measurement-based entanglement between two electron spins separated by 3 meters. By locally entangling each electron spin with a photon and performing a subsequent joint measurement of the photons, entanglement between the electron spins is heralded. The generated Bell-pair shared by remote locations is then used to unconditionally teleport the state of a nuclear spin in one diamond to the electron spin in the other diamond. To this end the nuclear spin is prepared in the state that is to be teleported followed by a local measurement of the electron and nuclear spin in the Bell-basis. This measurement projects the electron spin in the other diamond to the initial state of the nuclear spin up to a unitary operation that depends on the outcome of the Bell-state measurement. The original state is then recovered via a feed-forward operation. The fact that the protocol to prepare the remote Bell-pair is heralded and that the local Bell-state measurement can distinguish all four Bell-states, allows for unconditional teleportation. \\

The final two chapters of this thesis discuss the use of weakly coupled $^{13}$C spins as a quantum memory that is robust against optical excitation of the electron spin. The ability to store a quantum state in a quantum register while remotely connecting it to other registers is a prerequisite to implement entanglement purification and quantum repeater protocols. A theoretical model is introduced to analyze the dephasing of a carbon spin during repetitive resets of the electron spin (which is required in the presented heralded entanglement protocol). This model is tested in an isotopically purified sample with a carbon spin that is relatively insensitive to pertubations induced by electron spin-flips owing to its low coupling strength (200 Hz). Although the observed dephasing is stronger than predicted by the model the results indicate that it is possible to store a quantum state in the $^{13}$C spin while optically exciting the electron spin.

\chapter{Samenvatting}

