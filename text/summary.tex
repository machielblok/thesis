\chapter{Summary}

Gaining precise control over quantum systems is crucial for developing quantum technologies like quantum information processing and quantum sensing and to perform experimental tests of quantum mechanics. The experiments presented in this thesis implement quantum measurements and real-time feedback protocols that can help to achieve these goals using single electron and nuclear spins in diamond. Spins associated with the Nitrogen Vacancy (NV) center in diamond recently emerged as an excellent testbed to demonstrate quantum effects and are a promising building block for future quantum technology.\\

The NV center is an atomic defect in the diamond lattice consisting of a substituional nitrogen atom next to an empty lattice site. With its effective electron spin and nearby nuclear spins it forms a natural multi-qubit register with long-lived spin states that can be manipulated with magnetic resonance techniques. At temperatures below 10 K it displays spin-selective optical transitions that can be individually addressed and thereby provide an optical interface enabling high-fidelity single-shot readout and the generation of spin-photon entanglement.\\

In chapter 3 the fundamental trade-off between information and disturbance associated with a quantum measurement is investigated. A variable strength measurement of the host nitrogen spin is implemented via an indirect measurement using the electron spin. Here the measurement strength can be tuned by varying the amount of entanglement between the two spins. To avoid dephasing of the nuclear spin due to spin-flips during the electron spin readout a dynamical-stop readout is used to perform a QND measurement of the electron spin. This enables sequential partial measurements that can manipulate the nuclear spin using only the back-action of quantum measurement combined with real-time feedback. \\

NV centers can sense static magnetic fields using repetitive Ramsey sequences. The experiments presented in chapter 4 address the open question whether adaptive measurements can out-perform non-adaptive protocols for sensing. An adaptive strategy is implemented where the readout basis is optimized in real-time using a bayesian estimation based on previous measurement outcomes. The experiment shows that this adaptive protocol outperforms the best known non-adaptive protocol when overhead is taken into account.\\

The results in chapter 5 demonstrate measurement-based entanglement between two electron spins separated by 3 meters. By locally entangling each electron spin with a photon and subsequently performing a joint measurement of the photons entanglement between the electron spins is heralded.

\chapter{Samenvatting}