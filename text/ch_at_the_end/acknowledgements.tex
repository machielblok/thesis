%\documentclass{report}
%\begin{document}
\chapter{Acknowledgements}

\textit{``I do not want to end up like one of those scientists, obsessed with his/her research that is performed in a dark laboratory in some basement.''}
\\

This is what I said eleven years ago after visiting the open day of Delft University of Technology. I had decided to enroll in the BSc program for Applied Physics because I loved physics in high school, but clearly there was a slight prejudice against scientific research. Ironically, it was exactly in such a dark lab in the basement that I changed my mind. I specifically remember one of the first days of my Msc research project when my supervisor, Gijs, told me  ``Common, lets rotate a single spin by hand''. We were watching the light emitted by a single electron trapped in a diamond sample and as we rotated a knob to generate different magnetic fields we could see the signal drop as we hit the sweet spot where the magnetic field was on resonance with the spin of the electron. In essence we were manually operating one of the worlds smallest and most precise MRI machines.
Of course this is just one example and I consider myself lucky to have met so many inspiring people during my time in Delft who have motivated and supported me throughout my PhD.\\

Ronald, you have been an amazing supervisor and it has been a privilege to work in your group. Thank you for giving me the freedom to develop and pursue my own research interests while also providing guidance on how to conduct science. You taught me a lot of valuable things perhaps sometimes unconsciously, both scientific and non-scientific. Thanks to you I will always aim to efficiently use my measurement time and appreciate the importance of oscillations and amplifiers that go to eleven. I am amazed by the apparent ease with which you lead your group. You trust your students to do their experiments, yet are always up-to-date and ready to give advice when needed. At the same time you and the PI's at the quantum transport group have created excellent conditions to perform high-impact experiments, while keeping everyone focused by, as you called it, `protecting students from external noise'. \\
\newpage
I want to thank my colleagues from Team Diamond for all the fun we had both in the lab and outside. With the increasing experimental complexity all our projects rely on team work and I really enjoyed the collaborative atmosphere in our group. Cristian, thank you for being my partner in crime for the adaptive measurement experiments. I hope that you can soon start your own group. Hannes, Wolfgang and Bas, it was a pleasure to generate remote entanglement and perform teleportation with you, although I can not say that I will miss the overnight shifts to keep the experiment going. Hannes, our fabrication sessions in the cleanroom were always productive, both for making samples and for good conversation. Also, I would not have taken up Korean lessons without you, thanks for that. Thank you Wolfgang, for being very patient in teaching me how to code in python. Bas, I will never forget our marathon debug session of the Hermite pulses or our little projects like running teleportation on our phones from the train and cooling our drinks with liquid nitrogen. After almost 10 years of mostly overlapping careers perhaps it is time that we do something different ;) I`m sure you will find a great postdoc position in Australia or elsewhere. Tim, thank you for the countless times that a short question evolved into an insightful discussion. I really learned a lot from you and wish you all the best with your new group.

Toeno, thanks for the great time I had with you measuring on the room-temperature setup during my master project. Gijs, whether it was at the QT-coffee table or in the middle of a party, we always had great discussions which usually converged towards the measurement problem. One day I hope we can discuss the actual solution, for now I want to thank you for the good times we had and for coming over from the USA to be my paranimf. Lucio, thank you for introducing me to the low-temperature setup, I had a lot of fun opening quantum boxes with you. Julia, thanks for many science and non-science related conversations. Keep up the good job on conducting and communicating science. 

A lot of exciting experiments lie ahead on the Team Diamond roadmap. Suzanne, Stefan, Anais, Norbert and Andreas, it has been great working with you and I'm looking forward to reading about cavities, down-conversion, quantum repeaters and fundamental tests of quantum mechanics in the future. Also thanks to the many master students in Team Diamond Arthur, Oleksiy, Just, Ron, Adriaan, Gerwin, Anne Marije, Michiel, Abou, Sten and Arian. Koen it was a lot of fun to supervise you during your master thesis project. You did a great job on managing a complicated setup on your own and finding a carbon spin to control in the purified sample, which was the basis of the results presented in the final chapter of this thesis. I also had the pleasure to supervise Lisanne and Arno during their Bsc project. \\

I have had great technical support on many aspects of the experiments. Raymond (sr. and jr.) and Marijn, thank you for all advice on the microwave side of the experiment and the fast logic devices that enabled the real-time feedback. Bram, Jelle, Remco, Mark a warm thanks to you for the many vessels of Helium that kept us cold. From TNO I want to thank Remco for working together on the deformable mirror. For all administrative questions I could always turn to Yuki, Marja and Simone, thanks for the great help there. I've had several opportunities to present my work to a broader audience, thank you Michel and Heera for many useful tips and for motivating me to do so.\\

Of course I want to thank everybody in QT and, more recently, QuTech. I think it is quite unique to have a large group, with such a diversity of experiments, that maintains a strong interaction also known as the QT-family-feeling. Thanks to Leo K, Leo DC, Lieven, Stephanie and Ronald, for creating and expanding this inspiring research environment. For me it was highly motivating to keep in touch with research from other groups. Leo DC, I really learned a lot from talking to you about ancilla-based measurements. And Sal, it feels like you are also part of QT so I'll thank you here for discussing a wide range of physics topics and much more. Enjoy the rest of your PhD and I hope that you will soon resolve single phonons with your drum.

It is very exciting to watch QuTech grow and I'm curious to see how it will develop in the coming years. While QT/QuTech stands for excellent research, I think that part of its success comes from the great atmosphere outside the lab. I have warm memories of the many QT events, parties and uitjes, taking over the dance floor of a local bar, performing with the QT band and watching late-night wrestling. Many thanks to hotel-Pfaff for providing a much appreciated place to stay and thanks to Stefan, Daniel and \"{O}nder for continuing this tradition. Vincent, probably the man with the highest Mourik-index in the world, thanks for many laughs and discussions on the difference between fundamental science and engineering. Leo K thanks for attending my first-year progress meetings and our conversations about academia. And to all other Bsc-, Msc-, PhD-students, postdocs and staff: thank you for a great time in QT and QuTech! \\

To all my friends and family (in law), thanks for occasionally asking a question about quantum mechanics (you know I'm very happy to explain), for many kind reactions on the coverage of my research in popular media and for understanding when I sometimes skipped soccer-practice, jaarclub etentjes, bestuurs uitjes or drinks to be in the lab. A special thanks to my parents Tonny, Peter and to Marijk for all the love and support that you have given me. And to my lovely wife Esther, thank you for always being there, for giving me the best reason to leave the lab and come home, for being my best friend and for agreeing to name our cat after a famous physicist. 

\begin{flushright}
Machiel Blok\\
Delft, 2015
\end{flushright}
%\end{document}