\graphicspath{{./ch_introduction/figures/}}

\chapter{Introduction}
\label{ch:intro}

\begin{center} 
    \vspace{-1cm} {M.S. ~Blok} 
\end{center}

This is an exciting time to be a quantum physicist since we are in the midst of what has been called `the second quantum revolution'.
In the beginning of the 20$^{th}$ century, the first quantum revolution introduced a new way to describe our world at the smallest scale with very counter-intuitive consequences. Quantum mechanics predicts that elementary particles such as electrons and photons behave like waves that can be in two places at the same time (superposition) and cannot be observed without being perturbed (collapse of the wavefunction). Many physicists found these concepts hard to grasp since they contradict our everyday observations and Erwin Schr\"{o}dinger tried to solve this paradox by stating: 
\\
\\
\textit{``We never experiment with just one electron or atom or (small) molecule. In  thought-experiments we sometimes assume that we do; this invariably entails ridiculous consequences… we are not experimenting with single particles, any more than we can raise Ichtyosauria in the zoo.''}
\\

\section{The Second Quantum Revolution}
Since its development numerous experiments have verified the surprising features of quantum mechanics in a variety of systems such as single photons, trapped atoms and ions, superconducting circuits and single spins in semiconductors or color centers. With the increasing experimental control over quantum systems, efforts are shifting from testing quantum mechanics towards using it in new technologies: The second quantum revolution.

Perhaps the most well-known example of quantum technology is the quantum computer which performs calculations with hardware based on two-level quantum systems called quantum bits or qubits. Like the bits in a classical computer these quantum bits are represented by a logical 0 or 1, but unlike classical bits, qubits can also be in a superposition state: representing 0 and 1 at the same time. As a result the degrees of freedom that can be represented simultaneously for a system of \textit{N} qubits grows exponentially as $2^N$. Richard Feynman realized that this property could be used to simulate complex systems in nature that are incomputable even with the fastest modern computers. Roughly a decade later the first quantum algorithms were introduced, predicting an exponential speed-up in factorizing large numbers and a quadratic speed-up for searching unsorted data.

Another exciting idea is to connect remote locations via entangled quantum states to built a quantum network. This could facilitate the scaling up of small quantum processors to a larger quantum computer. Furthermore it will enable secure communication since the encryption of information sent over the quantum internet can be guaranteed by the laws of quantum mechanics.

Since quantum systems are extremely sensitive to interactions with the environment they can also be employed for precision measurements. Quantum sensors based on single spins for instance can measure temperature, and electric-, and magnetic-fields at the nanoscale, while atomic clocks allow for accurate frequency measurements.

The development of quantum technology is still in an early stage and most of the proof-of-principle experiments to date involve only passive control. However, small perturbations to quantum states can be detrimental and therefore many future applications will require active stabilization of the system. The focus of this thesis is to develop robust quantum measurements and active feedback protocols for quantum information and sensing.
\cite{Taylor_NatPhys_2008}
\section{Diamonds are for quantum}

\clearpage


\bibliographystyle{../thesis}
\bibliography{Thesis_introduction}
