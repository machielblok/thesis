\graphicspath{{./ch_introduction/figures/}}

\chapter{Introduction}
\label{ch:intro}

\begin{center} 
    \vspace{-1cm} {M.S. ~Blok} 
\end{center}

This is an exciting time to be a quantum physicist since we are in the midst of what has been called `the second quantum revolution'.
In the beginning of the 20$^{th}$ century, the first quantum revolution introduced a new way to describe our world at the smallest scale with very counter-intuitive consequences. Quantum mechanics predicts that elementary particles such as electrons and photons behave like waves that can be in two places at the same time (superposition) and cannot be observed without being perturbed (collapse of the wavefunction). Many physicists found these concepts hard to grasp since they contradict our everyday observations and Erwin Schr\"{o}dinger tried to solve this paradox by stating: 
\\
\\
\textit{``We never experiment with just one electron or atom or (small) molecule. In  thought-experiments we sometimes assume that we do; this invariably entails ridiculous consequences… we are not experimenting with single particles, any more than we can raise Ichtyosauria in the zoo.''}
\\

\section{The Second Quantum Revolution}


\clearpage


\bibliographystyle{../thesis}
\bibliography{intro}
